\PassOptionsToPackage{unicode=true}{hyperref} % options for packages loaded elsewhere
\PassOptionsToPackage{hyphens}{url}
\PassOptionsToPackage{dvipsnames,svgnames*,x11names*}{xcolor}
%
\documentclass[]{article}
\usepackage{lmodern}
\usepackage{amssymb,amsmath}
\usepackage{ifxetex,ifluatex}
\usepackage{fixltx2e} % provides \textsubscript
\ifnum 0\ifxetex 1\fi\ifluatex 1\fi=0 % if pdftex
  \usepackage[T1]{fontenc}
  \usepackage[utf8]{inputenc}
  \usepackage{textcomp} % provides euro and other symbols
\else % if luatex or xelatex
  \usepackage{unicode-math}
  \defaultfontfeatures{Ligatures=TeX,Scale=MatchLowercase}
\fi
% use upquote if available, for straight quotes in verbatim environments
\IfFileExists{upquote.sty}{\usepackage{upquote}}{}
% use microtype if available
\IfFileExists{microtype.sty}{%
\usepackage[]{microtype}
\UseMicrotypeSet[protrusion]{basicmath} % disable protrusion for tt fonts
}{}
\IfFileExists{parskip.sty}{%
\usepackage{parskip}
}{% else
\setlength{\parindent}{0pt}
\setlength{\parskip}{6pt plus 2pt minus 1pt}
}
\usepackage{xcolor}
\usepackage{hyperref}
\hypersetup{
            pdftitle={Sample size for prediction of quantitative and binary outcomes based on cohort study},
            pdfauthor={ZWu},
            colorlinks=true,
            linkcolor=Maroon,
            filecolor=Maroon,
            citecolor=Blue,
            urlcolor=blue,
            breaklinks=true}
\urlstyle{same}  % don't use monospace font for urls
\usepackage[margin=1in]{geometry}
\usepackage{color}
\usepackage{fancyvrb} 
\newcommand{\VerbBar}{|}
\newcommand{\VERB}{\Verb[commandchars=\\\{\}]}
\DefineVerbatimEnvironment{Highlighting}{Verbatim}{commandchars=\\\{\}}
% Add ',fontsize=\small' for more characters per line
\usepackage{framed}
\definecolor{shadecolor}{RGB}{248,248,248}
\newenvironment{Shaded}{\begin{snugshade}}{\end{snugshade}}
\newcommand{\AlertTok}[1]{\textcolor[rgb]{0.94,0.16,0.16}{#1}}
\newcommand{\AnnotationTok}[1]{\textcolor[rgb]{0.56,0.35,0.01}{\textbf{\textit{#1}}}}
\newcommand{\AttributeTok}[1]{\textcolor[rgb]{0.77,0.63,0.00}{#1}}
\newcommand{\BaseNTok}[1]{\textcolor[rgb]{0.00,0.00,0.81}{#1}}
\newcommand{\BuiltInTok}[1]{#1}
\newcommand{\CharTok}[1]{\textcolor[rgb]{0.31,0.60,0.02}{#1}}
\newcommand{\CommentTok}[1]{\textcolor[rgb]{0.56,0.35,0.01}{\textit{#1}}}
\newcommand{\CommentVarTok}[1]{\textcolor[rgb]{0.56,0.35,0.01}{\textbf{\textit{#1}}}}
\newcommand{\ConstantTok}[1]{\textcolor[rgb]{0.00,0.00,0.00}{#1}}
\newcommand{\ControlFlowTok}[1]{\textcolor[rgb]{0.13,0.29,0.53}{\textbf{#1}}}
\newcommand{\DataTypeTok}[1]{\textcolor[rgb]{0.13,0.29,0.53}{#1}}
\newcommand{\DecValTok}[1]{\textcolor[rgb]{0.00,0.00,0.81}{#1}}
\newcommand{\DocumentationTok}[1]{\textcolor[rgb]{0.56,0.35,0.01}{\textbf{\textit{#1}}}}
\newcommand{\ErrorTok}[1]{\textcolor[rgb]{0.64,0.00,0.00}{\textbf{#1}}}
\newcommand{\ExtensionTok}[1]{#1}
\newcommand{\FloatTok}[1]{\textcolor[rgb]{0.00,0.00,0.81}{#1}}
\newcommand{\FunctionTok}[1]{\textcolor[rgb]{0.00,0.00,0.00}{#1}}
\newcommand{\ImportTok}[1]{#1}
\newcommand{\InformationTok}[1]{\textcolor[rgb]{0.56,0.35,0.01}{\textbf{\textit{#1}}}}
\newcommand{\KeywordTok}[1]{\textcolor[rgb]{0.13,0.29,0.53}{\textbf{#1}}}
\newcommand{\NormalTok}[1]{#1}
\newcommand{\OperatorTok}[1]{\textcolor[rgb]{0.81,0.36,0.00}{\textbf{#1}}}
\newcommand{\OtherTok}[1]{\textcolor[rgb]{0.56,0.35,0.01}{#1}}
\newcommand{\PreprocessorTok}[1]{\textcolor[rgb]{0.56,0.35,0.01}{\textit{#1}}}
\newcommand{\RegionMarkerTok}[1]{#1}
\newcommand{\SpecialCharTok}[1]{\textcolor[rgb]{0.00,0.00,0.00}{#1}}
\newcommand{\SpecialStringTok}[1]{\textcolor[rgb]{0.31,0.60,0.02}{#1}}
\newcommand{\StringTok}[1]{\textcolor[rgb]{0.31,0.60,0.02}{#1}}
\newcommand{\VariableTok}[1]{\textcolor[rgb]{0.00,0.00,0.00}{#1}}
\newcommand{\VerbatimStringTok}[1]{\textcolor[rgb]{0.31,0.60,0.02}{#1}}
\newcommand{\WarningTok}[1]{\textcolor[rgb]{0.56,0.35,0.01}{\textbf{\textit{#1}}}}
\usepackage{graphicx,grffile}
\makeatletter
\def\maxwidth{\ifdim\Gin@nat@width>\linewidth\linewidth\else\Gin@nat@width\fi}
\def\maxheight{\ifdim\Gin@nat@height>\textheight\textheight\else\Gin@nat@height\fi}
\makeatother
% Scale images if necessary, so that they will not overflow the page
% margins by default, and it is still possible to overwrite the defaults
% using explicit options in \includegraphics[width, height, ...]{}
\setkeys{Gin}{width=\maxwidth,height=\maxheight,keepaspectratio}
\setlength{\emergencystretch}{3em}  % prevent overfull lines
\providecommand{\tightlist}{%
  \setlength{\itemsep}{0pt}\setlength{\parskip}{0pt}}
\setcounter{secnumdepth}{0}
% Redefines (sub)paragraphs to behave more like sections
\ifx\paragraph\undefined\else
\let\oldparagraph\paragraph
\renewcommand{\paragraph}[1]{\oldparagraph{#1}\mbox{}}
\fi
\ifx\subparagraph\undefined\else
\let\oldsubparagraph\subparagraph
\renewcommand{\subparagraph}[1]{\oldsubparagraph{#1}\mbox{}}
\fi

% set default figure placement to htbp
\makeatletter
\def\fps@figure{htbp}
\makeatother


\title{Sample size for prediction of quantitative and binary outcomes based on
cohort study}
\author{ZWu}
\date{28 November, 2021}

\begin{document}
\maketitle

For EPPIC\_2021 application (PI Jean King. Project/Core PIs: Korkin,
Ruiz).

Consider a cohort study of three groups: MBSR, acupuncture, and control,
each has same sample size n. The total sample size is 3n.

\hypertarget{predict-quantiative-outcome-based-on-regression-model-assumptions}{%
\section{Predict quantiative outcome based on regression model
assumptions}\label{predict-quantiative-outcome-based-on-regression-model-assumptions}}

Sample size for pediction accuracy of quantiative outcomes based on
simulations.

Consider a linear mixed effect model as the true underlying model:
\[Y_{ij} = \eta_{ij}(X) + \epsilon_{ij}\], where
\[\eta_{ij}(X) = \beta_0 + b_i + \beta_{acup}*acup_{ij} + \beta_{mbsr}*mbsr_{ij} + \beta_{sex}*sex_{ij} + \beta_{age}*age_{ij} + \beta_{edu}*edu_{ij} + \beta_{len}*len_{ij} + \beta_{base}*base_{ij} + \beta_{conc}*conc_{ij} + \sum_{k=1}^p\beta_k x_{kij} + \sum_{l=1}^L \gamma_l*acup_{ij}*z_{lij} + \sum_{m=1}^M \gamma_m*mbsr_{ij}*z_{mij}.\]
and
\[b_i \sim N(0, \sigma^2_i) \quad \text{ and } \quad \epsilon_{ij} \overset{iid}{\sim} N(0, \sigma^2).\]

Interpretation:

\begin{itemize}
\item $Y_{ij}$: quantitative response variable of pain reduction (i.e., TreatmentImpact, after a given period of time since the beginning of treatment) for the $j$th individual in the $i$th racial group. Positive value means pain reduces. 
\item $\beta_0$: a "natural" pain reduction corresponds to no-treatment (no acupuncture nor MBSR) and all other covariates being 0.
\item $b_i \sim N(0, \sigma^2_i)$: The mixed-effect (clustering effect) of the $i$th racial group. Individuals in the same racial group have the same $b_i$ value so that their pair-wise covariance is $\sigma^2_i$.  

\item $acup = 1$ if acupuncture treatment, $=0$ o.w. $\beta_{acup}$ is the effectof acupuncture. 
\item $mbsr = 1$ if MBSR treatment, $=0$ o.w. $\beta_{mbsr}$ is the effectof MBSR.
\item $sex = 1$ if male, $=0$ if female. $\beta_{sex}$ is the effect of sex. 
\item $age$: standardized age value (mean 0 and sd 1). $\beta_{age}$ is the effect of age. 
\item $edu$: education level: 0 / 1, 50\% each. 
\item $len$: duration length of pain. short or long: 0 / 1, 50\% each. 
\item $bas$: Baseline pain score. standardized. 
\item $conc$: presence of certain concomitant diseases: 0/1, 50\% each. 

\item $x_k$: standardized value of the $k$th PainMarker, $k=1, ..., p$. $\beta_k$'s are the corresponding effects. They are from OMICS- and biopsychosocial PainMarkers. 
\item $z_{lij}$ and $z_{mij}$ are modifiers for acupuncture and MBSR, respectively. They are from OMICS- and biopsychosocial PainMarkers.

\item The error SD $\sigma$ can be used to adjust the relative effects (or signal-to-noise ratio) regarding $\beta$ parameters, and the variation explained model ($R^2$). 

\item Cohen's $$f^2=R^2/(1-R^2)=\frac{\sigma^2_i + \beta_{acup}^2Var(acup) + \beta_{mbsr}^2Var(mbsr) + \beta_{sex}^2Var(sex) + \beta_{age}^2Var(age)  + \sum_{k=1}^p\beta_k^2 Var(x_k)}{\sigma^2}$$, where R^2 is the coefficient of determination, the proportion of the variation in the dependent variable that is predictable from the independent variable(s). 
\end{itemize}

Considerations:

\begin{itemize}
\item $b_i$: consider 3 groups, roughly equal numbers in the sample.
\item $\beta_{acup}$, $\beta_{mbsr}$, $\beta_{sex}$, and $\beta_{sex}$ are used to set/control their $R^2$, i.e., the percentage of variation explained by these "basic" factors. 
\item $\beta_{sex}$: Consider males are easier to reduce pain than females. Assume 50% recruited are males.
\item $\beta_k$, $\gamma_l$, $\gamma_m$ are used to set the percentage of variation explained by these extra factors.

\item Interaction terms are for Hypothesis 1 saying that the effects of treatments are "modulated by biopsychosocial factors".  
\item This study does address the aim of clustering patients ("Clustering and discovery of EPPIC-TreatmentPhenotypes" in Project 3). If some factors/markers are positively and some are negatively interacted with acupuncture/MBSR, then we could decide which treatment is better for them based on their markers. [Check Hong's project for AbbVie on subgrouping patients for drug treatment.]  
\end{itemize}

\hypertarget{predict-binary-outcome-based-on-logistic-model-assumptions}{%
\section{Predict binary outcome based on logistic model
assumptions}\label{predict-binary-outcome-based-on-logistic-model-assumptions}}

Consider a generalized linear mixed effect model as the true underlying
model:

\[E(P(Y_{ij}=1|X)) = \frac{1}{1+exp(-\eta_{ij}(X))},\] where \(X\) is
the matrix of covariate data, and \(\eta_{ij}(X)\) is same as above
(except that the coefficient values could be different)

Interpretation:

\begin{itemize}
\item $Y_{ij}$: binary response variable of TreatmentResponse for the $j$th individual in the $i$th racial group. $1$ for pain relief, $0$ for no relief. 
\item $\beta_0$: the baseline pain relieve probability when no interventions and all other covariates being 0. Together with othe terms, we can adjust $\beta_0$ to control the prevalence of $Y$.  
\item We considered the ORs of some predictors based on Tables 2 and 3 of \cite{witt2011patient} "Patient characteristics and variation in treatment outcomes: which patients benefit most from acupuncture for chronic pain?"
\end{itemize}

Considerations:

We use AUC to measure how well the factors explains / contributes to the
model. AUC and Cox and Snell R2 are connected
(\url{https://onlinelibrary.wiley.com/doi/full/10.1002/sim.8806}). Both
are used to represent how well the factors explains the model (Cox and
Snell R2 is an extension of R2 for measuring the percentage of variation
explained in regression). Both are interchangeably used for calculating
the minimum sample size based on the criterion regarding Nagelkerke's
R-squared value. See The formula by given in Fig 5 of paper
\cite{riley2020calculating}:
\url{https://www.research.manchester.ac.uk/portal/files/161373531/bmj.m441.full.pdf}

\begin{Shaded}
\begin{Highlighting}[]
\NormalTok{a=}\DecValTok{1}\NormalTok{;}
\CommentTok{#  }
\CommentTok{#   library(MASS)}
\CommentTok{#   library(nlme);}
\CommentTok{#   source("/Users/zheyangwu/ResearchDoc/Computation/CodingLibraries/myRLibrary/Prediction/Lib_Prediction.R");  }
\CommentTok{#   source("/Users/zheyangwu/ResearchDoc/Computation/CodingLibraries/myRLibrary/Simulations/Simulate_Data/Lib_simu_genetic_data.R"); }
\CommentTok{#   source("/Users/zheyangwu/ResearchDoc/Computation/CodingLibraries/myRLibrary/Prediction/generate_response_formula.R");   }
\CommentTok{# }
\CommentTok{# }
\CommentTok{# }
\CommentTok{# }
\CommentTok{# ###Parameter setting}
\CommentTok{# }
\CommentTok{#   ###Parameters on the effects / coefficients}
\CommentTok{#   beta0 = -0.5; #For binary trait, the risk at X=0 is 1/(1+exp(-beta0)). The average risk (prevalence) at given coefficients and predictors can be calculated using the code from function get.Y.logit: First set beta0=0, then run the code to get Xmatrix, then set XData=Xmatrix, then use the code in get.Y.logit to get Xbeta, then choose beta0's value such that mean(Y) reach the wanted prevalence. }
\CommentTok{#                      # Y = array(NA, length(Xbeta));}
\CommentTok{#                      #  for (i in 1:length(Y)) \{}
\CommentTok{#                      #    Py1x = 1/(1 + exp(-Xbeta[i]) - beta0); # P(y=1 | x) based on logistic model}
\CommentTok{#                      #    Y[i] = ifelse(runif(1) < Py1x, 1, 0);}
\CommentTok{#                      #  \}}
\CommentTok{#                      #  mean(Y); #prevalance of Y.}
\CommentTok{#  }
\CommentTok{#   #beta0 = -7; # for binary trait. Risk when X=0 is 1/(1+exp(7))=0.0009110512. Also, at the given coefficients and factors, beta0 = -7 makes the prevalence of Y roughly 0.5. }
\CommentTok{#   #beta0 = -3; # for binary trait. Risk when X=0 is 1/(1+exp(3))=0.04742587. Also, at the given coefficients and factors, beta0 = -3 makes the prevalence of Y, i.e., mean(Y), roughly 0.7. }
\CommentTok{# }
\CommentTok{# }
\CommentTok{#   raceN = 3; #Number of racial groups}
\CommentTok{#   sigma.race = 1; #pair-wise covariance among individuals in the same racial group}
\CommentTok{#   beta.race = 1; #treat the "coefficient" of the mix-effect b_i be 1. }
\CommentTok{# }
\CommentTok{#   beta.acup = log(4.9); #6; #Coeff of acupuncture}
\CommentTok{#   beta.mbsr = log(4.9); #6; #Coeff of mbsr}
\CommentTok{#   beta.sex = log(1.1); #1;}
\CommentTok{#   beta.age = log(1.25); #-1; }
\CommentTok{#   beta.edu = log(1.26); #1;}
\CommentTok{#   beta.len = log(1.13); #1;}
\CommentTok{#   beta.bas = log(0.80);#1;}
\CommentTok{#   beta.conc = log(0.77); #1;}
\CommentTok{# }
\CommentTok{#   markerN = 10; #The # of PainMarkers}
\CommentTok{#   beta.biom.v = 2; #=1; #The value of the coeff of the biomarkers}
\CommentTok{#   beta.biom = rep(beta.biom.v, markerN); #Coeff of the biomarkers}
\CommentTok{#   names.xbiom = paste("xbiom", 1:markerN, sep=""); #variable names for biomarkers}
\CommentTok{# }
\CommentTok{#   mdfN.acup = 4; #The # of modifiers for acupuncture}
\CommentTok{#   beta.mdf.acup.v=2; #The value of the coeff of accupunctur's modifiers}
\CommentTok{#   beta.mdf.acup = rep(beta.mdf.acup.v, mdfN.acup); #Vector of coeffs of accupunctur's modifiers }
\CommentTok{#   gama.acup.v = 2; #The value of the coeff of the modifier*acupucture interaction terms.}
\CommentTok{#   gama.acup = rep(gama.acup.v, mdfN.acup); #Vector of coeffs of the modifier*acupucture interaction terms.}
\CommentTok{# }
\CommentTok{#   names.mdf.acup = paste("zmdfAcup", 1:mdfN.acup, sep=""); #variable names of the modifiers for acupuncture}
\CommentTok{# }
\CommentTok{#   mdfN.mbsr = 4; #The # of modifiers for MBSR}
\CommentTok{#   beta.mdf.mbsr.v=2; #The value of the coeff of MBSR's modifiers}
\CommentTok{#   beta.mdf.mbsr = rep(beta.mdf.mbsr.v, mdfN.mbsr); #Vector of coeffs of accupunctur's modifiers }
\CommentTok{#   gama.mbsr.v = 2; #The value of the coeff of the modifier*MBSR interaction terms.}
\CommentTok{#   gama.mbsr = rep(gama.mbsr.v, mdfN.mbsr); #Vector of coeffs of the modifier*MBSR interaction terms.}
\CommentTok{# }
\CommentTok{#   names.mdf.mbsr = paste("zmdfMBSR", 1:mdfN.mbsr, sep=""); #variable names of the modifiers for MBSR}
\CommentTok{# }
\CommentTok{#   errSD=1; #The SD of error term }
\CommentTok{# }
\CommentTok{# ###Parametters on prodiction process}
\CommentTok{#   isRandomCV=T; #Random cross-validation in prediction }
\CommentTok{#   nfold=5; #The number of folds in cross-validation}
\CommentTok{#   nrepeat=2; #number of repeats of cross-validation}
\CommentTok{# }
\CommentTok{# ###Parameters on simulations}
\CommentTok{#   simuN = 100; #The number of simulations.}
\CommentTok{# }
\CommentTok{# ###Data simulation and prediction outcomes}
\CommentTok{#   predProp = c(0, 0.25, 0.5, 0.75, 1); #Proportion of true predictors besides names.basic that are included in prediction model. }
\CommentTok{#   models = vector(mode = "list", length(predProp)); #prediction models .}
\CommentTok{#   outputs = vector(mode = "list", length(predProp)); #prediction outputs .}
\CommentTok{# }
\CommentTok{# ###Parameters on data}
\CommentTok{# groupSampleSizes = seq(20, 400, by=20);}
\CommentTok{# #groupSampleSizes = seq(500, 2000, by=100);}
\CommentTok{# #groupSampleSizes = seq(150, 500, by=50);}
\CommentTok{# AUC = array(NA, dim=c(length(groupSampleSizes), length(predProp))); #Store AUC over sample sizes and proportions of predictors used. }
\CommentTok{# for(gi in 1:length(groupSampleSizes))\{}
\CommentTok{#   groupSampleSize = groupSampleSizes[gi]; #sample size for each of the three groups: control, mbsr, and acupuncture. }
\CommentTok{# }
\CommentTok{# ###Looping through simulations}
\CommentTok{#   R2.controls = array(NA, simuN); #Store the R2 of the controlling predictors.}
\CommentTok{#   for(i in 1:simuN) \{}
\CommentTok{#     ###Generate data}
\CommentTok{#     x0 = rep(1, groupSampleSize*3);}
\CommentTok{#     }
\CommentTok{#     #The mixed-effect term for racial group }
\CommentTok{#     xrace = sample(1:raceN, size=groupSampleSize*3, replace=T, prob=rep(1/raceN, raceN));}
\CommentTok{#       #Assume equal chance for each racial group to be sampled. }
\CommentTok{#     b.xrace = array(NA, dim=groupSampleSize*3); #b.xrace is the vector of b_i values. }
\CommentTok{#     for (racei in 1:raceN)\{ b.xrace[which(xrace==racei)] = rnorm(1, sd=sigma.race); \}}
\CommentTok{#       #assign the same b_i value for the all in the ith racial group. }
\CommentTok{#     }
\CommentTok{#     #The "basic" factors}
\CommentTok{#     xacup = c(rep(0, groupSampleSize*2), rep(1, groupSampleSize)); #acupuncture group indicator}
\CommentTok{#     xmbsr = c(rep(0, groupSampleSize), rep(1, groupSampleSize), rep(0, groupSampleSize)); #MBSR group indicator}
\CommentTok{#     xsex = rbinom(n=groupSampleSize*3, size=1, prob=0.5); #50% recruited are males??}
\CommentTok{#     xage = rnorm(n=groupSampleSize*3, mean=0, sd=1); #standardized age. }
\CommentTok{#     xedu = rbinom(n=groupSampleSize*3, size=1, prob=0.5);}
\CommentTok{#     xlen = rbinom(n=groupSampleSize*3, size=1, prob=0.5);}
\CommentTok{#     xbas = rnorm(n=groupSampleSize*3, mean=0, sd=1);}
\CommentTok{#     xconc = rbinom(n=groupSampleSize*3, size=1, prob=0.5);}
\CommentTok{#     }
\CommentTok{#     #PainMarker data}
\CommentTok{#     xbiom = data.frame(matrix(rnorm(n=groupSampleSize*3*markerN), ncol=markerN)); #Assume biomarker values are N(0,1);}
\CommentTok{#     names(xbiom) = names.xbiom; }
\CommentTok{#     }
\CommentTok{#     #acupuncture-modifier data}
\CommentTok{#     zmdfAcup = data.frame(matrix(rnorm(n=groupSampleSize*3*mdfN.acup), ncol=mdfN.acup)); #Assume acupuncture-modifier values are N(0,1);}
\CommentTok{#     names(zmdfAcup) = names.mdf.acup;}
\CommentTok{# }
\CommentTok{#     #mbsr-modifier data}
\CommentTok{#     zmdfMBSR = data.frame(matrix(rnorm(n=groupSampleSize*3*mdfN.mbsr), ncol=mdfN.mbsr)); #Assume acupuncture-modifier values are N(0,1);}
\CommentTok{#     names(zmdfMBSR) = names.mdf.mbsr;}
\CommentTok{#         }
\CommentTok{#     Xmatrix = cbind(x0, b.xrace, xacup, xmbsr, xsex, xage, xedu, xlen, xbas, xconc, xbiom, zmdfAcup, zmdfMBSR);}
\CommentTok{#     }
\CommentTok{#     #### Generate response }
\CommentTok{#     names.mainEff  = c("x0", "b.xrace", "xacup", "xmbsr", "xsex", "xage", "xedu", "xlen", "xbas", "xconc", names.xbiom, names.mdf.acup, names.mdf.mbsr);}
\CommentTok{#     coeffs.mainEff = c(beta0, beta.race, beta.acup, beta.mbsr, beta.sex, beta.age, beta.edu, beta.len, beta.bas, beta.conc, beta.biom, beta.mdf.acup, beta.mdf.mbsr);}
\CommentTok{#     names.trt = c("xacup", "xmbsr"); #variable names of the treatments}
\CommentTok{#     names.mdf = list(names.mdf.acup, names.mdf.mbsr); #variable names of the modifiers corresponding to the treatments. Require: Each element is a verctor of modifier names corresponding to the treatment in names.trt.}
\CommentTok{#     coeffs.interaction = list(gama.acup, gama.mbsr); #coefficients of the treatment-modifier interaction terms. Require: Consistent with names.trt and names.mdf.}
\CommentTok{#     }
\CommentTok{#     }
\CommentTok{#     # ####----Quantitative Response-----}
\CommentTok{#     # resp = get.Y.reg(XData=Xmatrix, names.mainEff, coeffs.mainEff, names.trt, names.mdf, coeffs.interaction, errSD=errSD);}
\CommentTok{#     # #print(resp$R2); #proportion of variation explained by all predictors}
\CommentTok{#     # }
\CommentTok{#     # #Calculate the R^2 of the controlling predictors}
\CommentTok{#     # vars.control = c(varBeta0=0, varRace=sigma.race^2, varAcup=(1/3)*(1-1/3), varMbsr=(1/3)*(1-1/3), varSex=(1/2)*(1-1/2), varAge=1, varEdu=.25, varLen=0.25, varBas=1, varConc=0.25); #, varBiom=rep(1, markerN));}
\CommentTok{#     # betas.control = c(beta0, beta.race, beta.acup, beta.mbsr, beta.sex, beta.age, beta.edu, beta.len, beta.bas, beta.conc); #, beta.biom, beta.mdf.acup, beta.mdf.mbsr);}
\CommentTok{#     # R2.controls[i] = sum(vars.control*betas.control^2)/var(resp$Y);}
\CommentTok{# }
\CommentTok{#     ####----Binary Response-----}
\CommentTok{#     resp = get.Y.logit(XData=Xmatrix, names.mainEff, coeffs.mainEff, names.trt, names.mdf, coeffs.interaction);}
\CommentTok{#     }
\CommentTok{#     ####Combine data for analysis }
\CommentTok{#     xrace = as.factor(xrace); #Convert race into factor variable, which is used in data analysis.}
\CommentTok{#     dat = data.frame(Y=resp$Y, Xmatrix, xrace);}
\CommentTok{#     }
\CommentTok{#     }
\CommentTok{#     ####Predictive analysis}
\CommentTok{#     names.control = c("xrace", "xacup", "xmbsr", "xsex", "xage", "xedu", "xlen", "xbas", "xconc"); #Controlling factors in prediction models}
\CommentTok{#     names.trt.all =  c("xacup", "xmbsr"); #All possible treatments that could have interaction effects. }
\CommentTok{#     for (mi in 1:length(predProp))\{}
\CommentTok{#       ##Create model formula based on proportion of predictors used. }
\CommentTok{#       xbiom.used    = round(length(names.xbiom)*predProp[mi]);}
\CommentTok{#       mdf.acup.used = round(length(names.mdf.acup)*predProp[mi]);}
\CommentTok{#       mdf.mbsr.used = round(length(names.mdf.mbsr)*predProp[mi]);}
\CommentTok{#       names.main=c(names.control, names.xbiom[0:xbiom.used], names.mdf.acup[0:mdf.acup.used], names.mdf.mbsr[0:mdf.mbsr.used]);}
\CommentTok{#       if (predProp[mi]==0) \{}
\CommentTok{#         names.trt = NULL;}
\CommentTok{#       \} else\{}
\CommentTok{#         names.trt = names.trt.all;}
\CommentTok{#         names.mdf = list(names.mdf.acup[0:mdf.acup.used], names.mdf.mbsr[0:mdf.mbsr.used]);}
\CommentTok{#       \}}
\CommentTok{#       models[[mi]]= formula.f.r(names.main=names.main, names.trt=names.trt, names.mdf=names.mdf, names.random=c("1"), names.group=c("xrace"));}
\CommentTok{#     }
\CommentTok{#       # }\AlertTok{###}\CommentTok{ ---- Predict quantitative outcome ----}
\CommentTok{#       # #out = meanPredEvaluCV.lme(fixed=models[[mi]]$fixed, dat=dat, randomf=models[[mi]]$random, model_R='lme', predLevel=1, loopn=nrepeat, cvNumber=nfold); #Prediction by lme. }
\CommentTok{#       # out = meanPredEvaluCV.lme(fixed=models[[mi]]$fixed, dat=dat, randomf=NULL, model_R='lm', loopn=nrepeat, cvNumber=nfold); #Prediciton by lm. }
\CommentTok{#       # outputs[[mi]] = rbind(outputs[[mi]], t(c(MSE=out[1], L2normRatio=out[2], L1normRatio=out[3], correlation=out[4], MSEoverObsVar=out[5])));}
\CommentTok{#       }
\CommentTok{#       }
\CommentTok{#       }\AlertTok{###}\CommentTok{ ---- Predict quantitative outcome ----}
\CommentTok{#       out = predEvaluCV.glm(formula=models[[mi]]$fixed, dat=dat, nfold=nfold, nrepeat=nrepeat, isRandomCV=isRandomCV);}
\CommentTok{#       #outputs[[mi]] = rbind(outputs[[mi]], t(c(prob=out[1], sensi=out[2], speci=out[3], AUC=out[4])));}
\CommentTok{#       outputs[[mi]] = rbind(outputs[[mi]], t(unlist(out)));}
\CommentTok{#     }
\CommentTok{#     \}}
\CommentTok{#   \}}
\CommentTok{#   }
\CommentTok{#   # #True underlying model}
\CommentTok{#   print(c(sampleSize=groupSampleSize*3, raceN=raceN, sigma.race = sigma.race,  beta.acup = beta.acup,}
\CommentTok{#     beta.mbsr = beta.mbsr, beta.sex = beta.sex, beta.age = beta.age, beta.edu=beta.edu, beta.len=beta.len, beta.bas=beta.bas, beta.conc=beta.conc, markerN = markerN, beta.biom.v = beta.biom.v,}
\CommentTok{#     mdfN.acup=mdfN.acup, beta.mdf.acup.v=beta.mdf.acup.v, gama.acup.v=gama.acup.v,}
\CommentTok{#     mdfN.mbsr=mdfN.mbsr, beta.mdf.mbsr.v=beta.mdf.mbsr.v, gama.mbsr.v=gama.mbsr.v));}
\CommentTok{#   }
\CommentTok{#   # #R2 of controlling predictors}
\CommentTok{#   # mean(R2.controls); }
\CommentTok{#   }
\CommentTok{#   #Prediction accuracies}
\CommentTok{#   for (mi in 1:length(predProp))\{}
\CommentTok{#     # print(predProp[mi]);}
\CommentTok{#     # print(models[[mi]]$fixed);}
\CommentTok{#     # print(apply(outputs[[mi]], 2, mean));}
\CommentTok{#     # print(apply(outputs[[mi]], 2, quantile, probs=c(0.05, 0.95)));}
\CommentTok{#     }
\CommentTok{#     AUC[gi, mi] = apply(outputs[[mi]], 2, mean)[4];}
\CommentTok{#   \}}
\CommentTok{# \} }
\CommentTok{# cbind(groupSampleSizes*3, AUC); }
\CommentTok{#     }
\end{Highlighting}
\end{Shaded}

\end{document}
