\chapter{Discussion}

The present study utilized the analytical result to examine the relationship between $pPMSEr$, sample size $n$, and $Cohen's f^2$. Our findings indicate that there is a significant relationship between these variables, with $pPMSEr$ decreasing as sample size and effect size increase. These results are consistent with previous research on statistical power and highlight the importance of adequate sample sizes in hypothesis testing.
The analysis also revealed that the non-basic predictors have the ability to significantly contribute to explaining the variability in response. The variance of the response variable, pain intensity, was assumed to be 1 in the simulation as the error variance for the full model is 0.9393, which implies that the model was able to capture a substantial amount of the variance in pain intensity.
The effects of each predictor were also examined using the calculated coefficients. The results indicated that Physical functioning, Depressive symptoms, and LOC-internal were the most significant predictors of pain intensity. These findings are consistent with previous research, which has also identified these factors as important contributors to pain intensity.

It is worth noting that our results of predictor effects differ from those reported by \cite{baker2008chronicpain}, who utilized the same correlation matrix but different coefficients. While this discrepancy may be due to methodological differences or variations in sample characteristics, it underscores the importance of replication studies in scientific research.

One limitation of our study is that it only considered a single quantitative outcome variable (pain intensity) and a limited set of predictors, without considering the interactions or random effects. Future research could extend our approach to other kinds of models such as general linear models, additional outcome measures, and a more comprehensive set of predictors, potentially utilizing machine learning or other statistical techniques to identify complex relationships among predictors. Other possible limitation includes the use of a correlation matrix rather than real-world data and the assumption of normality in the distribution of the response variable and covariates. These limitations may affect the accuracy of the results and should be taken into consideration when interpreting the findings.


Future research may benefit from using individual-level data and exploring the relationship between response and other factors obtained in newly published studies. Moreover, additional studies may also investigate the effectiveness of interventions aimed at reducing pain intensity by addressing the significant predictors identified in this study.


