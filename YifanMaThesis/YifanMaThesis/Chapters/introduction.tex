\chapter{Introduction}

\noindent Predictive modeling is playing an increasingly important role in many fields, including finance, healthcare, and marketing. However, achieving high predictive accuracy is a complex task that depends on multiple factors, including the number of predictors and their effect size, as well as the sample size used to train and test the model. 

The focus of this thesis is to investigate the relationship between prediction accuracy, the number of predictors, their effect sizes, and the sample size. Our goal is to establish recommendations for designing studies that aim to enhance a model's predictive power by identifying new predictors. To accomplish this, we will measure the impact of various factors that contribute to the PMSE.

The importance of sample size in predictive modeling has been recognized for several decades. In 1974, \cite{narula1974predictive}explored the relationship between predictive mean square error and stochastic regressor variables. Since then, numerous studies have been conducted to determine the optimal sample size for different types of outcomes and model structures. For instance, \cite{riley2019minimum}, \cite{riley2019minimums}published the minimum sample size required for developing multivariable prediction models for continuous, binary, and time-to-event outcomes. \cite{van2014modern} conducted a simulation study to determine the sample size requirements for predicting dichotomous endpoints, and \cite{van2019sample} discussed the limitations of the events per variable criteria for sample size determination in binary logistic prediction models. These studies have contributed to our understanding of the importance of sample size in predictive modeling and provided valuable insights into determining the optimal sample size based on the outcome of interest and model structure. 

While a larger sample size can lead to more accurate predictions, there is a limit to the effect of sample size on model performance. One reason for this limit is that beyond a certain point, adding more samples may not provide much additional information. For example, if a dataset already contains representative samples of the population, collecting more samples may not significantly improve the model's predictive ability. Another reason is that the relationship between the input variables and the target variable may not be linear or may have a limited range. In such cases, adding more data may not improve model performance beyond a certain point. Additionally, the complexity of the model and the quality of the features used in the model can also impact its predictive ability.  It is essential to acknowledge that the true predictive model remains unknown and that achieving 100\% prediction accuracy may not be possible, even with knowledge of the effect sizes of the predictors. The uncertainty of the error term can still have an impact on the model's accuracy, which is why it's essential to consider other factors in addition to sample size to ensure high prediction accuracy.

Therefore, while the sample size is an important consideration in predictive modeling, its effect on model performance is limited, and other factors such as model complexity and feature quality also play a crucial role. On the other hand, new predictors can provide additional information about the outcome variable, potentially capturing previously unknown or unmeasured factors that are related to the outcome. This can help to increase the model's predictive power and provide a more comprehensive understanding of the factors that contribute to the outcome. In practical medical reports, the effect size of newly added predictors usually only considers this new variable and controls the effect size of other existing variables.


This thesis aims to investigate the relationship between prediction accuracy and newly developed predictors considering sample size. Through analyzing the impact of various factors on PMSE, we will develop guidelines for study designs that can improve the accuracy and reliability of research results. We will not only investigate the relationship between sample size, number of predictors, and effect sizes on the PMSE but also summarize the degree of influence of new predictors on the model's prediction accuracy. To quantify this influence, we will use the reduced prediction mean square error percentage $rPMSEp$ or Correlation between true value and prediction, which corresponds to the efficient sample size required to achieve a certain prediction accuracy. We will provide calculations based on the pain study to support our findings.